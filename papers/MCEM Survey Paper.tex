\documentclass[11pt, oneside]{article}   	% use "amsart" instead of "article" for AMSLaTeX format
\usepackage{geometry} 
\usepackage{color}               		% See geometry.pdf to learn the layout options. There are lots.
\geometry{letterpaper}                   		% ... or a4paper or a5paper or ... 
%\geometry{landscape}                		% Activate for rotated page geometry
%\usepackage[parfill]{parskip}    		% Activate to begin paragraphs with an empty line rather than an indent
\usepackage{graphicx}				% Use pdf, png, jpg, or eps§ with pdflatex; use eps in DVI mode
								% TeX will automatically convert eps --> pdf in pdflatex		
\usepackage{amssymb}
\usepackage{setspace}

\usepackage[title]{appendix}   % Start an appendices environment, then treat each separate appendix as an ordinary section
								% [title] changes labels to, e.g., "Appendix A...". Omit to label as "A...".

\usepackage{authblk}

\usepackage{mathrsfs}	% For \mathcal{}, a script font in math mode

\usepackage[semicolon]{natbib}
\usepackage{verbatim}
\usepackage{soul}	% Highlighting
\usepackage{graphicx}
\usepackage{subcaption}
\usepackage{hyperref}
\def\UrlBreaks{\do\/\do-}


\usepackage{multirow}

\usepackage{esdiff} %Shorter syntax for derivatives. Use \diff{f}{x} or \diffp{f}{t}
\usepackage{amsmath}
\usepackage{amsthm}

\usepackage{enumitem}   % For more customizable lists

\usepackage{etoolbox}   % Prerequisite for imakeidx
\usepackage{imakeidx}   % For making an index
\makeindex              % Initialize the index (this command must be included!)


% \usepackage[ruled]{algorithm2e} % For writing algorithms

\usepackage{algpseudocode}
\usepackage{algorithm} % For writing algorithms

\usepackage{glossaries}
\makeglossaries



\newcommand{\lt}{LTHC}
\newcommand{\llt}{\ell(\theta)}


\newcommand{\bF}{\mathbb{F}}
\newcommand{\bG}{\mathbb{G}}
\newcommand{\bP}{\mathbb{P}}
\newcommand{\bQ}{\mathbb{Q}}
\newcommand{\bV}{\mathbb{V}}
\newcommand{\bE}{\mathbb{E}}
\newcommand{\bR}{\mathbb{R}}

\newcommand{\iid}{\overset{\mathrm{iid}}{\sim}}
\DeclareMathOperator*{\argmax}{argmax}
\DeclareMathOperator*{\argmin}{argmin}

\newcommand{\cd}{f_c(y, x; \theta)}
\newcommand{\od}{f(y; \theta)}

\newcommand{\hq}{\hat{Q}}

\newcommand{\wcirc}{\mathring{w}}

\newcommand{\mh}{Metropolis-Hastings}


\newtheorem{proposition}{Proposition}[section]



\newglossaryentry{densities} %! This doesn't actually work. Inline glossary entries are like citations, they print the name of the entry. I could instead separate each density into its own glossary entry, but this will lead to a large, poorly organized glossary.
{
    name={$f, f_c, f_m$},
    description={Observed data, complete data, missing data densities (or PMFs)}
}


\title{\SARS\ Transmission in University Classes}
\author[1]{William Ruth}
\author[2]{Richard Lockhart}
\affil[1]{Corresponding Author - Department of Statistics and Actuarial Science \\ Simon Fraser University \\ Burnaby, BC  Canada \\ wruth@sfu.ca}
\affil[2]{Department of Statistics and Actuarial Science \\ Simon Fraser University \\ Burnaby, BC  Canada}
%\affil{Department of Statistics and Actuarial Science \\ Simon Fraser University \\ Burnaby, BC  Canada \\ lockhart@sfu.ca}
%\date{\today}							% Activate to display a given date or no date
\date{}

\begin{document}
%\maketitle

% \doublespacing

\begin{abstract}
    We survey the EM algorithm and its Monte Carlo-based extensions.
\end{abstract}

\begin{itemize}
    \color{red}
    \item Replace ``conditional distribution of the missing data given the observed data'' with ``missing data distribution''
    \item \citet{Caf05} use $M_k$ for the Monte Carlo size at iteration $k$. This is a useful definition for discussing other papers too.
    \item I have changed my iteration labels. $\hat{\theta}_{k-1}$ is now the maximizer of the current MCEM objective function, and $\hat{\theta}_{k-1}$ was used to construct that objective function. Watch for this when editing.
    \item Parameter space has not yet been defined.
    \item Is the ``a'' in EM algorithm capitalized or not?
\end{itemize}

\section{The EM Algorithm}

%todo - Define observed, complete and missing data distributions
%todo - Define parameter space.

We define three distributions which will be central to our study of missing data problems. Let $Y$ be the observed data and $X$ be the missing data. Note that $X$ need not correspond to any actual real-world process, but may instead be a conceptual device which facilitates analysis of the data which were actually observed. We refer to the distribution of $Y$ as the ``observed data distribution''\index{Observed Data Distribution}, and write $f$ for its density (or mass function). We refer to the joint distribution of $Y$ and $X$ as the ``complete data distribution''\index{Complete Data Distribution}, and write $f_c$ for its density. We refer to the conditional distribution of $X$ given $Y$ as the ``missing data distribution''\index{Missing Data Distribution}, and write $f_m$ for its density. Note that the missing data distribution is not the marginal distribution of the missing data, but rather its conditional distribution given the observed data.

The EM algorithm is a method for analyzing incomplete data which was formalized by \citet{Dem77}. See \citet{McL08} for an excellent book-length overview of the EM algorithm. We begin by discussing a probabilistic framework within which the EM algorithm is often applied. We then present the EM algorithm in detail. Finally, we discuss some limitations of this method. Throughout, we illustrate our presentation with a toy problem based on linear regression with a single, unobserved, covariate.

The EM algorithm consists of iterating two steps. First is the expectation, or ``E'', step, in which an objective function is constructed from the complete data likelihood. Second is the maximization, or ``M'', step, in which the previously computed objective function is maximized. These two steps are then alternated until some convergence criterion is met. Whatever value of $\theta$ the algorithm converges to is used as our parameter estimate. We now go into more detail on each of the two steps.

The E-step of the EM algorithm is where we construct the objective function which will be used to update our parameter estimate. This objective function is the conditional expectation of the complete data likelihood, given the observed data. If our complete data can be partitioned into an observed component, $Y$, and a missing component, $X$, then our objective function at iteration $k$ is given by
%
\begin{align}
    Q(\theta|\theta_{k-1}) & = \bE_{\theta_{k-1}}[\ell_c(\theta; y, X) | Y=y]
\end{align}
%
where $\ell_c$ is the log-likelihood of the complete data model. Note that the conditional expectation uses our parameter estimate from the previous iteration.

The M-step of the EM algorithm consists of maximizing the objective function constructed in the previous E-step. That is, we define $\theta_k = \argmax\limits_\theta Q(\theta|\theta_{k-1})$. Typically, this optimization must be performed numerically via, e.g., gradient ascent or Newton's method. See \citet{Noc06} for details and other optimization algorithms. 

We can combine the E- and M-steps of the EM algorithm into a single ``update function''. We write $M(\theta_{k-1}) = \argmax\limits_\theta Q(\theta|\theta_{k-1})$. The EM algorithm can thus be viewed as the iterative application of this update function, $M$.

\subsection{Properties}

\textbf{Section intro...}

\subsubsection{Ascent Property and Generalized EM}
\label{sec:GEM}

An important feature of the EM algorithm is its so-called ``ascent property''. This property says that an iteration of the EM algorithm (\hl{have I explicitly defined ``EM iteration''? Do I need to?}) never decreases the observed data likelihood. This is somewhat surprising, since EM updates are computed without ever evaluating the observed data likelihood. 

\begin{proposition}[Ascent Property of EM]
    Let $\theta \in \Theta$, and $\theta' = M(\theta)$ be the EM update from $\theta$. Then $\ell(\theta') \geq \ell(\theta)$.
\end{proposition}

\begin{proof}
    We begin by noting that the following decomposition holds for any value of $x$:
    %
    \begin{align}
        \ell(\theta; y) &= \ell_c(\theta; y, x) - \ell_m(\theta; y, x)
    \end{align}
    %
    Subtracting the values of both sides at $\theta$ from their values at $\theta'$ and taking conditional expectations, we get
    %
    \begin{align}
        \ell(\theta'; y) - \ell(\theta; y) &= Q(\theta'|\theta) - Q(\theta|\theta) + \bE_{\theta}[\ell_m(\theta; y, x) - \ell_m(\theta'; y, x)]\\
        &= Q(\theta'|\theta) - Q(\theta|\theta) + \mathrm{KL}(\theta || \theta') \label{eq:asc_KL}
    \end{align}
    %
    where the last term in line \ref{eq:asc_KL} is the Kullback-Leibler (KL) divergence from the missing data distribution with $\theta = \theta$ to the same distribution with $\theta = \theta'$. Note that KL divergences are always non-negative, so we get
    %
    \begin{align}
    \ell(\theta'; y) - \ell(\theta; y) &\geq Q(\theta'|\theta) - Q(\theta|\theta)    
    \end{align}
    %
    Finally, since $\theta'$ maximizes $Q(\cdot|\theta)$, we have $\ell(\theta'; y) - \ell(\theta; y) \geq 0$.
\end{proof}

In our proof of the ascent property, we only required that $Q(\theta'|\theta) \geq Q(\theta|\theta)$, not that $\theta'$ maximize $Q(\cdot|\theta)$. This observation leads to the definition of the ``Generalized EM Algorithm''\index{Generalized EM Algorithm}, which replaces the M-step with setting $\theta_k$ to any point in $\Theta$ such that $Q(\theta_k|\theta_{k-1}) \geq Q(\theta_{k-1}|\theta_{k-1})$.



\subsubsection{Recovering Observed Data Likelihood Quantities}

Under regularity conditions, it is possible to compute both the score vector and the observed information matrix of the observed data likelihood using complete data quantities. These regularity conditions consist of being able to interchange the order of differentiation and integration for various functions (\hl{awk? I wasn't feeling well when I wrote this.}). \hl{Does it make sense to define $\mathcal{I}_c$ and $\mathcal{I}_m$ in the following proposition?}

\begin{proposition}
    \label{thm:EM_decomp}
    The following identities hold (\hl{regularity conditions!}):
    \begin{enumerate}[label=(\roman*)]
        \item $S(\theta; y) = \bE_\theta [S_c(\theta; y, X)|Y=y]$ \label{eq:obs_score_identity}
        \item $I(\theta) = \mathcal{I}_c(\theta) - \mathcal{I}_m(\theta)$ \label{eq:obs_info_identity}\\
        where $\mathcal{I}_c(\theta) := - \bE_\theta \left[ \nabla^2 \ell_c(\theta; y, X) | Y=y \right]$ and $\mathcal{I}_m(\theta) := - \bE_\theta \left[ \nabla^2 \ell_m(\theta; y, X) | Y=y \right]$
    \end{enumerate}
\end{proposition}

\begin{proof}
    We start with expression \ref{eq:obs_score_identity}. Let $\Omega$ be the complete data sample space. Let $\mathcal{Y}$ and $\mathcal{X}$ be the observed and missing data sample spaces respectively. For every $y \in \mathcal{Y}$, let $\mathcal{X}(y) = \{ x \in \mathcal{X}: (y,x) \in \Omega\}$. Note that $f(y; \theta) = \int_{\mathcal{X}(y)} f_c(y, x; \theta) dx$.
    %
    \begin{align}
        \bE_\theta [S_c(\theta; y, X)|Y=y] &= \int_{\mathcal{X}(y)} \nabla \ell_c(\theta; y, x) f_m(y, x; \theta) dx \nonumber\\
        &= \int_{\mathcal{X}(y)} \frac{f_m(y, x; \theta)}{f_c(y, x; \theta)} \nabla f_c(\theta; y, x) dx \nonumber\\
        &= \int_{\mathcal{X}(y)} \frac{1}{f(y; \theta)} \nabla f_c(\theta; y, x) dx\nonumber\\
        &= \frac{1}{f(y; \theta)} \int_{\mathcal{X}(y)} \nabla f_c(\theta; y, x) dx\nonumber\\
        &= \frac{1}{f(y; \theta)} \nabla \int_{\mathcal{X}(y)} f_c(\theta; y, x) dx \nonumber\\
        &= \frac{1}{f(y; \theta)} \nabla f(y; \theta)\nonumber\\
        &= S(\theta; y) \nonumber
    \end{align}

    Proceeding now to \ref{eq:obs_info_identity}, we decompose the observed data log-likelihood as
    %
    \begin{align*}
        \ell(\theta; y) &= \ell_c(\theta; y, x) - \ell_m(\theta; y, x)
    \end{align*}
    %
    Differentiating twice and taking conditional expectations of both sides yields the required result.
\end{proof}

An alternative to Proposition \ref{thm:EM_decomp} part \ref{eq:obs_info_identity} which involves only conditional expectations of complete data quantities is given in the following proposition.

\begin{proposition}[Louis' Identity]
    \label{thm:info_decomp}
    Let $\hat{\theta}$ be a stationary point of the observed data log-likelihood. \hl{Assuming regularity conditions,} we can write the observed information of the observed data distribution at $\hat{\theta}$ as
    %
    \begin{align}
        I(\theta) = \mathcal{I}_c(\theta) - \bE_{\theta} [ S_c(\theta) S_c(\theta)^T | Y=y] + S(\theta) S(\theta)^T
    \end{align}
    %
    In particular, if $\hat{\theta}$ is a stationary point of the observed data log-likelihood, then
    %
    \begin{align}
        I(\hat{\theta}) = \mathcal{I}_c(\hat{\theta})  - \bE_{\hat{\theta}} [ S_c(\hat{\theta}) S_c(\hat{\theta})^T | Y=y] \label{eq:info_at_MLE}
    \end{align}
\end{proposition}

\begin{proof}
    We follow the derivation of \citet{Lou82}. For brevity, we write $f(\theta)$ and $f_c(\theta)$ for $f(y; \theta)$ and $f(y, x; \theta)$ respectively. Consider the following two Hessians:
    %
    \begin{align}
        \nabla^2 \ell(\theta) &= \nabla \left[ \int_{\mathcal{X}(y)} \frac{\nabla f_c(\theta) dx}{f(\theta)} \right]\\
        &= \int_{\mathcal{X}(y)} \frac{\nabla^2 f_c(\theta)}{f(\theta)} dx - \frac{1}{f(\theta)^2}\left( \int_{\mathcal{X}(y)} \nabla f_c(\theta) dx \right) \left( \int_{\mathcal{X}(y)} \nabla f_c(\theta) dx \right)^T\\
        &= \bE_\theta \left[ \left. \frac{\nabla^2 f_c(\theta)}{f_c(\theta)} \right| Y=y \right] - \bE_\theta \left[ \left. \frac{\nabla f_c(\theta)}{f_c(\theta)} \right| Y=y \right] \bE_\theta \left[ \left. \frac{\nabla f_c(\theta)}{f_c(\theta)} \right| Y=y \right]^T\\
        &= \bE_\theta \left[ \left. \frac{\nabla^2 f_c(\theta)}{f_c(\theta)} \right| Y=y \right] - S(\theta; y) S(\theta; y)^T \label{eq:hess_obs_lik}\\
        \nabla^2 \ell_c(\theta) &= \nabla \left( \frac{\nabla f_c(\theta)}{f_c(\theta)} \right)\\
        &= \frac{\nabla^2 f_c(\theta)}{f_c(\theta)} - S_c(\theta) S_c(\theta)^T \label{eq:hess_comp_lik}
    \end{align}
    %
    Combining lines \ref{eq:hess_obs_lik} and \ref{eq:hess_comp_lik}, we get
    %
    \begin{align}
        \nabla^2 \ell(\theta) &= \bE_\theta [ \nabla^2 \ell_c(\theta) | Y=y] + \bE_\theta [ S_c(\theta) S_c(\theta)^T | Y=y] - S(\theta; y) S(\theta; y)^T \label{eq:hess_obs_lik2}
    \end{align}
    %
    Finally, evaluating line \ref{eq:hess_obs_lik2} at $\theta = \hat{\theta}$ makes the rightmost term vanish, thereby yielding the required expression.
\end{proof}

Proposition \ref{thm:info_decomp} is known as Louis' standard error formula. Other decompositions for the observed information matrix of the observed data likelihood do exist; see, e.g., \citet{Oak99,McL08}. However, the one due to Louis will be most useful to us later.

\subsection{Example: Gene Frequency Estimation}
\label{sec:eg-genes}

Consider the problem of estimating allele frequencies based on observed phenotypes. Often, a single phenotype can be encoded by multiple genotypes with different configurations of dominant and recessive alleles. This is sometimes referred to as the problem of gene frequency estimation\index{Gene Frequency Estimation}. Our analysis closely follows Example 2.4 from \citet{McL08}\footnote{Although these authors do give data in their example, there is little context around this data, and I have been unable to locate more information from their references. I have opted instead to use my own dataset.}

We investigate a simplified model for blood type which consists of only the ABO blood group. See Chapter 5 of \citet{Dea05}, for a detailed introduction to blood types\index{Blood Type}. There are three alleles for this gene: A, B and O. Allele O is recessive, which alleles A and B exhibit co-dominance. That is, genotypes AO and AA encode blood type A, genotypes BO and BB encode blood type B, genotype OO encodes blood type O, and genotype AB encodes blood type AB. Suppose that we seek to estimate the proportion of each allele within a population, based on a sample of individuals' phenotypes. \citet{Fuj78} report blood types of 4,464,349 people in Japan collected between 1964 and 1975. This sample is so large that any standard errors are practically zero. To retain a reasonable level of uncertainty, we focus on a single administrative division, Oto, in Nara Prefecture. See Figure \ref{tab:blood_type} for details.

\begin{table}
    \centering
    \caption{Observed frequency and theoretical probability of each blood type \citep{Fuj78}}
    \begin{tabular}{c|cccc}
        Blood Type & O & A & B & AB\\
        \hline
        Random Variable & $Y_1$ & $Y_2$ & $Y_3$ & $Y_4$\\
        Observed Frequency & 10 & 16 & 7 & 1\\
        Probability & $r^2$ & $p^2 + 2pr$ & $q^2 + 2qr$ & $2pq$
    \end{tabular}
    \label{tab:blood_type}
\end{table}

Let $Y_1$, $Y_2$, $Y_3$ and $Y_4$ be the number of people with blood type O, A, B and AB respectively, and $Y = (Y_1, Y_2, Y_3, Y_4)$. Let $r$, $p$ and $q$ be the proportions of alleles O, A and B respectively within the population of interest. Since $r + p + q = 1$, we let $\theta = (p, q)$ be our target of inference. Pretending that the population size is fixed and that iid sampling was employed, $Y$ follows a multinomial distribution with $n =$ 4,464,349. Assuming homogeneous genetic mixing, the probability vector for $Y$ is $\pi = (r^2, p^2 + 2pr, q^2 + 2qr, 2pq)$.

Maximizing the likelihood in this model involves solving the score equation, a system of two 3rd-degree polynomials in $p$ and $q$. This can be done numerically, and gives estimates $p = 0.299$ and $q = 0.128$. These values match the ones given by \citet{Fuj78}. The information matrix and asymptotic covariance (inverse information matrix) are given by
%
\begin{align}
    I(\hat{\theta}) &= \begin{bmatrix}
        276 & 84.8\\
        84.8 & 584
    \end{bmatrix}\\
    \hat{\Sigma}_\mathrm{MLE} &= \begin{bmatrix}
        3.79 \cdot 10^{-3} & -5.49 \cdot 10^{-4}\\
        -5.49 \cdot 10^{-4} & 1.79 \cdot 10^{-3}
    \end{bmatrix}
\end{align}

\subsubsection{Complete Data}

The problem of gene frequency estimation would be much simpler if we could observe individuals' genotypes. We consider augmenting the observed data $Y$ by further classifying individuals by genotype. Let $X = (X_1, \ldots, X_6)$ be the genotypes of the individuals represented in Table \ref{tab:blood_type}. See Table \ref{tab:blood_type_complete}.

\begin{table}
    \centering
    \caption{Terminology and probabilities for our augmented version of the dataset in \citet{Fuj78}. We also give the blood type coded for be each genotype.}
    \begin{tabular}{c|cccccc}
        Genotype & OO & AO & AA & BO & BB & AB\\
        \hline
        Random Variable & $X_1$ & $X_2$ & $X_3$ & $X_4$ & $X_5$ & $X_6$\\
        Probability & $r^2$ & $2pr$ & $p^2$ & $2qr$ & $q^2$ & $2pq$\\
        Blood Type & O & A & A & B & B & AB
    \end{tabular}
    
    \label{tab:blood_type_complete}
\end{table}

Note that we can write $Y$ in terms of $X$. Specifically, $Y_1 = X_1$, $Y_2 = X_2 + X_3$, $Y_3 = X_4 + X_5$ and $Y_4 = X_6$. This corresponds to summing components of $X$ which correspond to the same blood type. The distribution of $X$ is multinomial, with the same sample size as $Y$, and probability vector given in Table \ref{tab:blood_type_complete}.

See Appendix \ref{app:blood_complete} for the complete data likelihood function and its derivatives.

\subsubsection{EM Algorithm}

The gene frequency estimation problem fits nicely into the EM algorithm framework. In this section, we present key quantities and results of our analysis. See Appendix \ref{app:blood}, especially part \ref{app:blood_miss}, for more details.

The EM objective function at iteration $k+1$ is
%
\begin{align}
    Q(\theta|\theta_k) &\equiv  \bE_{\theta_k}(n_O | y) \log r + \bE_{\theta_k}(n_A | y) \log p + \bE_{\theta_k}(n_B | y) \log q\\
    &=: \nu^{(k)}_O \log r + \nu^{(k)}_A \log p + \nu^{(k)}_B \log q
\end{align}
%
where $nu^{(k)}_O$, $\nu^{(k)}_A$ and $\nu^{(k)}_B$ are the expected number of O, A and B alleles respectively given $Y=y$ and $\theta = \theta_k$. Maximizing this objective function in $p$ and $q$ gives the following EM update function:
%
\begin{align}
    M(\theta_k) &= \begin{pmatrix}
        p_{k+1}\\
        q_{k+1}
    \end{pmatrix}\\    
    &= \begin{pmatrix}
        \nu^{(k)}_A / 2n\\
        \nu^{(k)}_B / 2n
    \end{pmatrix}
\end{align}
%
Starting with $\theta_0 = (1/3, 1/3)$ corresponding to equal proportions of the three alleles, Figure \ref{fig:blood_EM_traj} gives trajectories of the EM estimates for $p$ and $q$ using the data in Figure \ref{tab:blood_type}. These estimates converge quite quickly to the maximizer of the observed data likelihood.
%
\begin{figure}
    \centering
    \caption{Trajectory of EM estimates for $p$ and $q$ for the blood type example. Horizontal lines give the values of the MLE.}
    \label{fig:blood_EM_traj}
    \includegraphics[width=0.5\textwidth]{../plots/Blood_Type/EM_traj.pdf}
\end{figure}
%
Beyond computing the observed data MLE, we also need the standard error of this estimator. To this end, we compute the observed data information matrix using Louis' Method (see Proposition \ref{thm:info_decomp}). The asymptotic covariance matrix of our MLE is then the inverse of this information matrix. Omitting details (see Appendices \ref{app:blood_complete} and \ref{app:blood_miss}), both the observed data information matrix and asymptotic covariance match those obtained from the observed data likelihood. 
%


\section{The Monte Carlo EM Algorithm}
\label{sec:MCEM}

The Monte Carlo EM, or MCEM, algorithm was first proposed by \citet{Wei90}. This method proceeds by replacing the conditional expectation in the E-step of the EM algorithm with a Monte Carlo average. More precisely, at each iteration we generate observations from the conditional distribution of the missing data given the observed data, and average the complete data likelihood over this Monte Carlo sample. Formally, at a given iteration of the MCEM algorithm, let $X_1,\ldots, X_M$ be a Monte Carlo sample from the law of $X|Y=y$ with $\theta$ set to the value from the previous iteration, say $\theta_0$. Write
%
\begin{align}
    \hat{Q}(\theta|\theta_0) &= \sum_{i=1}^M w_i \ell_c(\theta; y, X_i)\\
    &:= \hat{\bE} \ell_c (\theta; y, X)
\end{align}
%
where the $w_i$ are sampling weights. \hl{Confirm that this operator notation isn't contradicted elsewhere.} Under iid sampling we simply get $w_i = M^{-1}$ for every $i$, but more intricate sampling schemes may have more complicated weights. The estimate of $\theta$ is then the maximizer of the MCEM objective function: $\hat{\theta} = \argmax_\theta \hat{Q}(\theta|\theta_0)$. Write $\hat{\theta}_{k-1}$ for the $k$th MCEM estimate. 

Provided that a valid sampling scheme is available for the missing data distribution, we can use Proposition \ref{thm:info_decomp} to estimate the observed data information matrix. 

\begin{proposition}
    Under the conditions of Proposition \ref{thm:info_decomp}, \hl{as well as any required for the sampler,} we get
    %
    \begin{align}
        - \hat{\bE}_{\hat{\theta}} \nabla^2 \ell_c(\hat{\theta}) - \hat{\bE}_{\hat{\theta}} S_c(\hat{\theta}) S_c(\hat{\theta})^T \rightarrow I(\theta)
    \end{align}
    %
    \hl{Under stronger conditions,} we also get asymptotic normality with variance obtained from importance sampling analysis.
\end{proposition}

The MCEM algorithm has the advantage of circumventing the challenge of computing potentially intractable conditional expectations for the EM algorithm. However, this analytical simplification does come at the cost of introducing some new computational problems. In this section, we outline the main problems faced by the MCEM algorithm and present various solutions which have been proposed in the literature. We focus primarily on practical aspects of the MCEM algorithm; see \citet{Nea13} for a survey of theoretical considerations.

Two problems which have received considerable attention in the literature are how to choose the Monte Carlo sample size at each iteration, and when to terminate the MCEM algorithm. These were identified early by \citet{Wei90}, but did not receive systematic treatment until later. We here give a brief overview of different authors' approaches to solving these two problems. The rest of this section goes into more detail on each of the methods. \citet{Wei90} suggest examining a plot of the parameter estimates across iterations, and either terminating or increasing the Monte Carlo size when the plot appears to stabilize. \citet{Cha95} use a pilot study to choose the Monte Carlo sample size, and terminate when a confidence interval for the improvement of the observed data log-likelihood between successive iterations contains zero. \citet{Boo99} frame each MCEM iteration as an M-estimation problem targeting the deterministic EM update. They increase the Monte Carlo size if an asymptotic confidence interval for the EM update contains the previous iteration's parameter estimate, and terminate when multiple consecutive iterations' estimates have sufficiently small relative error. \citet{Caf05} build confidence bounds for the increment in the EM objective function at each iteration of the MCEM algorithm. They increase the Monte Carlo size until the lower bound is positive and terminate when the upper bound is sufficiently small.

In the rest of this section, we give more detail on each of the implementations introduced above. We also illustrate each method on the blood type dataset described in Section \ref{sec:eg-genes}. The relevant conditional distribution and likelihood calculations are described in Appendix \ref{app:blood_miss}.

\subsection{Early Work (\citealp{Wei90})}

In their seminal work, \citet{Wei90} propose the MCEM algorithm and present a simple implementation. They illustrate that the complete data gradient and Hessian are easily obtained at each iteration from the Monte Carlo sample and, following \citet{Lou82}, give an estimator for the observed data information matrix. Regarding convergence, \citeauthor{Wei90} recommend plotting the parameter estimates across iterations and stopping when the estimates appear to stabilize around some constant. When this stabilization is detected, one can either declare convergence and stop, or increase the Monte Carlo size and continue iterating until the estimate trajectory again stabilizes.

In order to apply the MCEM algorithm to estimate allele frequencies in the blood type problem, we must specify the number of iterations, $K$, and the Monte Carlo size for each iteration, $M$. Starting conservatively, we use $K=50$ and $M=100$. Figure \ref{fig:blood_naive_MCEM_traj1} gives trajectories of the MCEM estimates of $p$ and $q$. These estimates appear to converge quickly to a stationary mean, but there is still some uncertainty around this mean. As such, we run MCEM for another 20 iterations with $M=1000$, staring with the final value from our first run. See Figure \ref{fig:blood_naive_MCEM_traj2}. The trajectories from our second run are much more stable around their means. We use the final values from these trajectories as our estimates: $\hat{p} = 0.298$ and $\hat{q} = 0.128$. These values closely match the maximizer of the observed data likelihood.

\begin{figure}
    \centering
    \caption{Trajectory of MCEM estimates of $p$ and $q$ for the blood type example. The horizontal lines correspond to maximum likelihood estimates.}
    \label{fig:blood_naive_MCEM_traj}
    \begin{subfigure}{0.45\textwidth}
        \centering
        \includegraphics[width = 1\textwidth]{../plots/Blood_Type/naive_MCEM_traj1.pdf}
        \caption{$M=100$}
        \label{fig:blood_naive_MCEM_traj1}
    \end{subfigure}
    \hfill
    \begin{subfigure}{0.45\textwidth}
        \centering
        \includegraphics[width = 1\textwidth]{../plots/Blood_Type/naive_MCEM_traj2.pdf}
        \caption{$M=1000$}
        \label{fig:blood_naive_MCEM_traj2}
    \end{subfigure}
   
\end{figure}

\subsection{Running a Pilot Study \citep{Cha95}}

Building on the ideas of \citeauthor{Wei90}, \citet{Cha95} develop a method for both choosing the Monte Carlo size and deciding when to terminate the MCEM algorithm. This method is based on an identity which allows us to estimate observed data likelihood ratios by Monte Carlo averages of complete data likelihood ratios. More precisely, we write
%
\begin{align}
	\frac{\mathcal{L}(\theta_1; y)}{\mathcal{L}(\theta_2; y)} &= \bE_{\theta_2} \left[ \left. \frac{\mathcal{L}_c(\theta_1; y, X)}{\mathcal{L}_c(\theta_2; y, X)} \right| Y=y \right] \label{eq:Cha95}
\end{align}
%
See \citet{Cha95} for a derivation. To apply Equation \ref{eq:Cha95} to the MCEM algorithm, we replace the conditional expectation on the right-hand side with a Monte Carlo average from the corresponding conditional distribution. This adjustment allows us to estimate log-likelihood ratios from the observed data distribution, without ever directly evaluating the observed data likelihood. 



The method of \citeauthor{Cha95} can be divided into two parts. The first part is a pilot study, in which we compute a standard error for our log-likelihood ratio estimator near the MLE, and determine what Monte Carlo size is required to get this standard error below a pre-specified threshold. In the second part, we increase the Monte Carlo size appropriately, and continue iterating until a confidence interval for the true observed data log-likelihood ratio contains zero. This two-part procedure reflects the suggestion of \citet{Wei90} to run MCEM until it appears to stabilize, then increase the Monte Carlo sample size to get a more precise estimate.


The pilot study portion of \citeauthor{Cha95}'s method proceeds with a fixed, ``moderately large'' Monte Carlo size.  In addition to tracking the parameter estimates across iterations, we also record the estimated log-likelihood ratio of the current estimate relative to the starting point of the algorithm. This ratio is computed by keeping a running cumulative sum of all one-step log-likelihood ratios. 

Note that the observed data log-likelihood ratio is estimated by a Monte Carlo average. In order to avoid bias, we do not use the same Monte Carlo sample to estimate this ratio as we used to compute the current parameter update. Instead, we generate the Monte Carlo sample which will be used to compute the next iteration's parameter update, and use this new sample to estimate the log-likelihood ratio\footnotemark. Since Monte Carlo variability is independent across the two iterations' samples \hl{(something, something, conditional independence)}, we remove any possible bias from our estimate of the log-likelihood ratio. We are then free to use this new Monte Carlo sample to compute the next iteration's parameter update.

\footnotetext{Due to our Monte Carlo sample using the new parameter estimate rather than the old, we must actually estimate the reciprocal of the likelihood ratio which we want, then multiply its log by $-1$. This is reflected in the formulas of \citeauthor{Cha95} but not discussed explicitly.}

Upon completion of the pilot study (i.e., after a pre-specified number of iterations), we select the estimate with maximal estimated observed data likelihood. We also select a few nearby estimates (\citeauthor{Cha95} suggest the 10 which follow the maximizer), and perform a small number of single-iteration MCEM runs from each of the chosen estimates. We compute the estimated observed data log-likelihood ratio for each of these MCEM updates, and get the variance of these updates for each parameter estimate. We then pool variances across parameter estimates into a standard error for our log-likelihood ratio estimator. Finally, we use this standard error, along with its known scaling rate with the Monte Carlo sample size\footnotemark, to determine what size is needed to get our standard error below the pre-specified threshold.

\footnotetext{Provided that the parameter estimates used to compute our standard error are close enough to the MLE, \citeauthor{Cha95} show that the standard error scales like $1/M$ rather than the usual $1 / \sqrt{M}$, where $M$ is the Monte Carlo size.}

With this newly determined Monte Carlo size, we return to the optimal parameter estimate from our pilot study and continue iterating the MCEM algorithm. At each step now, we also construct a confidence interval for the true observed data log-likelihood ratio corresponding to the current parameter update (i.e., between two consecutive parameter values, not from the starting point to the current estimate). We terminate the algorithm when such a confidence interval contains zero. This corresponds to no evidence of an improvement in the observed data likelihood.

Applying \citeauthor{Cha95}'s method to our blood-type dataset, we get the parameter estimate trajectory shown in Figure \ref{fig:blood_CL_MCEM_estimates}. Figure \ref{fig:blood_CL_MCEM_likelihoods} gives the trajectory of estimated observed data log-likelihood ratios relative to the starting point of the algorithm ($\hat{p}_0 = \hat{q}_0 = 1/3$), along with pointwise $95\%$ Wald-type confidence bands using the standard error computed after the pilot study and scaled appropriately for the Monte Carlo sample size. We use a stopping rule of $10^{-6}$. Note that this standard error only applies near the maximizer of the observed data MLE.

\begin{figure}
    \centering
    \caption{MCEM estimates of $p$ and $q$ for the blood type example, based on the method of \citet{Cha95}. The vertical dashed line shows the end of the pilot study. The horizontal lines correspond to maximum likelihood estimates.}
    \label{fig:blood_CL_MCEM_traj}
    \begin{subfigure}{0.45\textwidth}
        \centering
        \includegraphics[width = 1\textwidth]{../plots/Blood_Type/Chan_Ledolter_Traj.pdf}
        \caption{Estimate trajectories}
        \label{fig:blood_CL_MCEM_estimates}
    \end{subfigure}
    \hfill
    \begin{subfigure}{0.45\textwidth}
        \centering
        \includegraphics[width = 1\textwidth]{../plots/Blood_Type/Chan_Ledolter_lik_ratio_Traj.pdf}
        \caption{Estimated log-likelihood ratio trajectory}
        \label{fig:blood_CL_MCEM_likelihoods}
    \end{subfigure}
   
\end{figure}

\subsection{Uncertainty Quantification for the Parameter Estimate (\citealp{Boo99})}

 \citet{Boo99} use a somewhat different approach from either \citet{Wei90} or \citet{Cha95} to understand the behaiour of the MCEM algorithm. The method of \citeauthor{Boo99} is based on quantifying Monte Carlo uncertainty of the MCEM update as an approximation to the update which would be made by the deterministic EM algorithm from the same starting point. They then recommend starting the MCEM algorithm with a small Monte Carlo size, and adding more observations only when the parameter estimates are no longer changing discernibly across iterations. More formally, \citeauthor{Boo99} suggest building a confidence interval for the EM update based on the Monte Carlo variability of the MCEM update at each iteration. If this interval contains the previous iteration's parameter estimate, then the parameter updates are too small relative to the amount of Monte Carlo variability and more samples are required. These authors similarly recommend assessing convergence by checking for small relative error in the parameter updates. To account for the possibility of Monte Carlo variability leading to two consecutive estimates being similar before the algorithm has `converged', they suggest waiting until the relative error is small for three consecutive iterations.

The confidence interval used to quantify Monte Carlo uncertainty within an iteration is obtained by framing each step of the MCEM algorithm as the solution of an M-estimation problem. This allows us to inherit the desirable properties of M-estimators; specifically, asymptotic normality (see, e.g. \citealp{van98}). Following the usual M-estimator construction and assuming that the relevant regularity conditions hold, we are able to estimate the asymptotic variance of the MCEM parameter estimator at each iteration. Note that this standard error is based on the Monte Carlo variability within an iteration; it does not measure sampling variability due to the observed data. 

More formally, write $\tilde{\theta}_k$ for the EM update based on $\hat{\theta}_{k-1}$. Note that $\hat{\theta}_{k-1}$ is held fixed. \hl{Analysis of a single MCEM iteration is done conditional on the previous iteration (awk?).} Unless stated otherwise, all expectations are taken with $\theta = \hat{\theta}_{k-1}$. Assuming sufficient smoothness and moment conditions, we get the following expression for the MCEM update:
%
\begin{align}
    \sqrt{M}(\hat{\theta}_k - \tilde{\theta}_k) &= - \sqrt{M} \left[ \nabla^2 Q(\tilde{\theta}_k|\hat{\theta}_{k-1})\right]^{-1} \left[\nabla \hq(\tilde{\theta}_k|\hat{\theta}_{k-1}) \right] + o_p(1) \label{eq:th_as_dist}
\end{align}
%
where $M$ is the Monte Carlo size and $\nabla$ denotes differentiation with respect to the left argument of $Q$ or $\hat{Q}$. Note that the first expression on the right-hand side is the inverse Hessian of the EM objective function (fixed) while the second is the gradient of the MCEM objective function (an average). Thus, $\hat{\theta}_k$ is asymptotically normal with asymptotic variance
%
\begin{align}
    &\left[ \nabla^2 Q(\tilde{\theta}_k|\hat{\theta}_{k-1})\right]^{-1} \bV \left[ S_c (\tilde{\theta}_k) | Y=y \right] \left[ \nabla^2 Q(\tilde{\theta}_k|\hat{\theta}_{k-1})\right]^{-1}\\
    &\approx \left[ \nabla^2 \hat{Q}(\hat{\theta}_k|\hat{\theta}_{k-1})\right]^{-1} \hat{\bE} \left[ S_c(\hat{\theta}_k) S_c(\hat{\theta}_k)^T | Y=y \right] \left[ \nabla^2 \hat{Q}(\hat{\theta}_k|\hat{\theta}_{k-1})\right]^{-1} 
\end{align}
%
where $S_c$ is the complete data score vector, and $\hat{\bE}$ is the Monte Carlo average over the missing data, with $\hat{\theta}_k$ held fixed \hl{(remove comma?)}. Note that there is no first moment term in the conditional variance of $S_c$ because $\hat{\theta}_k$ is a maximizer of $\hat{\bE} [\ell_c |Y=y]$.

Based on the above discussion, we can build an asymptotic confidence interval for $\tilde{\theta}_k$, the EM update based on the MCEM estimate from iteration $k$. \citeauthor{Boo99} recommend checking whether this interval contains $\hat{\theta}_{k-1}$ and, if so, increasing $M$ for the next iteration. Specifically, they suggest starting the next iteration with $M/r$ more points, with $r = 3,4$ or $5$ working well in their examples \hl{(this whole discussion is pretty awkward)}.

To assess convergence of the MCEM algorithm, \citeauthor{Boo99} present two criteria. The first is a familiar measure of relative error in parameter estimates between consecutive iterations:
%
\begin{align}
    \max_j \left( \frac{\left| \hat{\theta}_{k, j} - \hat{\theta}_{k-1,j} \right|}{\left| \hat{\theta}_{k-1,j} \right| + \delta_1} \right) < \delta_2 \label{eq:Boo99_tol}
\end{align}
%
where $\delta_1$ and $\delta_2$ are small positive constants, and the subscript $j$ ranges over components of $\theta$. \citeauthor{Boo99} suggest using $\delta_1 = 10^{-3}$ and $\delta_2$ between $2 \cdot 10^{-3}$ and $5 \cdot 10^{-3}$. See \citep{need} (probably \citealp{Sea06}, p.\ 436, or \citealp{Mar63}) for why condition (\ref{eq:Boo99_tol}) has this particular form.

Alternatively, since \citeauthor{Boo99} apply their method to the analysis of generalized linear mixed models, where pathologies may arise due to parameter estimates being too close to a boundary, they propose a second stopping rule:
%
\begin{align}
    \max_j \left( \frac{\left| \hat{\theta}_{k, j} - \hat{\theta}_{k-1,j} \right|}{\sqrt{\bV \hat{\theta}_{k-1,j}} + \delta'_1} \right) < \delta'_2 \label{eq:Boo99_tol2}
\end{align}

\hl{How do we estimate the variance here?} The purpose of condition (\ref{eq:Boo99_tol2}) is to detect when estimated variance components are very close to zero and the numerical precision needed to satisfy condition (\ref{eq:Boo99_tol}) requires a prohibitive amount of computation.

We apply the method of \citet{Boo99} to our blood type example, with the settings recommended in their paper. Specifically, they suggest setting $\alpha = 0.25$, $k = 3$ (alternatively, $4$ or $5$), $\delta_1 = 0.001$, and $\delta_2 = 0.002$ (or as high as $0.005$). We also start with a Monte Carlo size of 10. Figure \ref{fig:blood_BH_traj} gives trajectories of the MCEM estimates, as well as the Monte Carlo size used to obtain each of these estimates. Note how the trajectory stabilizes as the MC size increases. The final estimate from this method is $\hat{p} = 0.299$, $\hat{q} = 0.128$; again, very close to the MLE.



\begin{figure}
    \centering
    \caption{Trajectory of estimates for $p$ and $q$, as well as Monte Carlo sample sizes, from the method of \citeauthor{Boo99}.}
    \label{fig:blood_BH_traj}
    \includegraphics[width=0.6\textwidth]{../plots/Blood_Type/Booth_Hobert_Traj.pdf}   
\end{figure}






\subsection{Uncertainty Quantification for the Objective Function \citep{Caf05}}

The approach of \citet{Caf05} is similar in spirit to that of \citet{Boo99}. Both sets of authors seek to quantify Monte Carlo uncertainty in the MCEM algorithm as an approximation to the EM algorithm. The difference is that where \citeauthor{Boo99} measure uncertainty in the parameter estimate, \citeauthor{Caf05} focus on uncertainty in the objective function. Specifically, \citeauthor{Caf05} base their analysis on asymptotic normality of the MCEM increment:
%
\begin{proposition}
    \label{thm:Caf_normality}
    Let $\Delta \hat{Q}(\hat{\theta}_k|\hat{\theta}_{k-1}) = \hat{Q}(\hat{\theta}_{k-1}|\hat{\theta}_{k-1}) - \hat{Q}(\hat{\theta}_k|\hat{\theta}_{k-1})$. Define $\Delta Q(\hat{\theta}_k|\hat{\theta}_{k-1})$ similarly. Let $M_k$ be the Monte Carlo size at iteration $k$. Then, \hl{assuming some regularity conditions},
    %
    \begin{align}
        \sqrt{M_k} \left[ \Delta \hat{Q}(\hat{\theta}_k|\hat{\theta}_{k-1}) - \Delta Q(\hat{\theta}_k|\hat{\theta}_{k-1}) \right] \rightsquigarrow N(0, \Sigma_k)
    \end{align}
    %
    As $M_k \rightarrow \infty$, where $\Sigma_k$ is an asymptotic covariance matrix.
\end{proposition}

\begin{proof}
    Following \citet{Caf05}, we write
    %
    \begin{align}
        \sqrt{M_k} \left[ \Delta \hat{Q}(\hat{\theta}_k|\hat{\theta}_{k-1}) - \Delta Q(\hat{\theta}_k|\hat{\theta}_{k-1}) \right] &= \sqrt{M_k} \left[ \Delta \hat{Q}(\theta_k|\hat{\theta}_{k-1}) - \Delta Q(\theta_k|\hat{\theta}_{k-1}) \right] \nonumber \\
        & + \sqrt{M_k} \left[ \Delta \hat{Q}(\hat{\theta}_k|\hat{\theta}_{k-1}) - \Delta Q(\theta_k|\hat{\theta}_{k-1}) \right]\\
        & + \sqrt{M_k} \left[ \Delta \hat{Q}(\theta_k|\hat{\theta}_{k-1}) - \Delta Q(\hat{\theta}_k|\hat{\theta}_{k-1}) \right]\\
        &=: A_k + B_k + C_k
    \end{align}
    %
    First, note that $A_k$ depends on the current Monte Carlo sample only through $\hat{Q}$, and is thus asymptotically normal by the ordinary Central Limit Theorem. However, $B_k$ and $C_k$ require a more careful analysis.

    \textbf{Use Taylor's Theorem when you're more awake.}
\end{proof}

Provided that we are able to estimate $\Sigma_k$, Proposition \ref{thm:Caf_normality} allows us to build asymptotic confidence intervals for the EM increment, $\Delta Q$. Recall that in Section \ref{sec:GEM}, we defined the Generalized EM algorithm by requiring that $\Delta Q \geq 0$, and showed that this requirement guarantees the ascent property. While the stochastic nature of the MCEM algorithm makes it impossible to guarantee that the EM increment is positive, we are able to use Proposition \ref{thm:Caf_normality} to construct asymptotic confidence bounds for $\Delta Q$. Provided that we can estimate $\Sigma_k$, we can then \hl{be reasonably confident that $\Delta Q > 0$ (awk)}.

Estimating the asymptotic variance under iid or importance sampling is fairly straightforward. Importance sampling however, is somewhat more complicated; particularly when a normalizing constant must be estimated. \citeauthor{Caf05} give a formula for importance sampling based on the Delta Method. They also give some guidance for calculating standard errors based on MCMC sampling, \hl{which we do not go into here. See Section XX for some details}. 

We now return to the key MCEM problems of choosing the Monte Carlo size and when to terminate. For the former, \citeauthor{Caf05} advise constructing a lower confidence limit for the EM increment, $\Delta Q$. If this limit is positive, then we proceed to the next iteration. If not, then we augment the Monte Carlo sample at the current iteration (with, say, $M_k/r$, with $r$ some small positive integer as in \citealp{Boo99}), and compute a new confidence bound. At the next iteration, \citeauthor{Caf05} advise using a starting Monte Carlo sample which is at least as large as the final sample from the previous iteration. \hl{In fact, a larger sample may be required based on extrapolating the MC variability from the previous iteration (clarify; I don't fully understand what they're doing here).}

\citeauthor{Caf05} base their termination criterion on stopping when there is evidence that the algorithm is no longer yielding sufficient improvement in the EM objective function. Specifically, they start by choosing a tolerance, $\tau>0$, then calculate an upper confidence limit for the EM increment at each iteration. If this upper confidence limit is below $\tau$, then we declare that there is little room for improvement left in the EM objective, and terminate our algorithm.

\begin{comment}
\subsection{Quantifying Monte Carlo Uncertainty}

In their seminal work, \citet{Wei90} highlight two important challenges with implementing this method: choosing the Monte Carlo sample size at each iteration, and deciding when to terminate the algorithm. It turns out that solutions to these two problems are often connected via their link to the uncertainty in our Monte Carlo conditional expectations. The recommendations given by \citeauthor{Wei90} on how to solve these problems are mostly informal, but much of the later work on the MCEM algorithm centers around developing more precise solutions \hl{(too strong of a statement?)}.

\citet{Cha95} present an alternative method for choosing the Monte Carlo sample size based on starting with a pilot study and using information near the optimal parameter estimate\footnotemark to choose a Monte Carlo size for the rest of the analysis. In contrast to \citet{Wei90}, \citeauthor{Cha95} use a fixed Monte Carlo size for their analysis.

\footnotetext{\citet{Cha95} present an identity which expresses the observed data log-likelihood ratio as the conditional expectation of the corresponding complete data log-likelihood ratio. Replacing the conditional expectation with a Monte Carlo average gives a natural estimate of the observed data log-likelihood ratio. This approach closely resembles the Monte Carlo Maximum Likelihood method of \citet{Gey94}. See \hl{Section ???}.}

The method of \citet{Boo99} centers on treating each iteration of the MCEM algorithm as an M-estimation problem targeting the deterministic EM update. This framework is quite natural, as an iteration of the MCEM algorithm consists of maximizing a Monte Carlo approximation to the EM objective function. Provided that certain regularity conditions are satisfied (see, e.g., \citealp{van98}), we can estimate the asymptotic standard error of the MCEM update as an estimate of the EM update with the same starting value, with the sampling variability induced by Monte Carlo simulation. \citeauthor{Boo99} then recommend constructing an asymptotic confidence set for the EM update, and increasing the Monte Carlo size if this confidence set contains the previous iteration's parameter estimate\footnotemark.

\footnotetext{For myself: As far as I can tell, in their paper, \citeauthor{Boo99} don't suggest that you check whether augmenting the Monte Carlo size produces a confidence set which does not contain the previous iteration's estimate. I think they're just suggesting you increase M and proceed with the next iteration. This differs from \citet{Caf05} who require that we keep checking and augmenting M until the condition is satisfied. \hl{I don't think this needs to go in the paper, but it's of some interest if you want to implement the method of} \citeauthor{Boo99}}
\end{comment}

\section{Alternatives to the MCEM Algorithm}
\label{sec:alternatives}

In this section, we outline some alternatives to the MCEM algorithm for maximizing the likelihood of an incomplete dataset. Examples include stochastic approximation \citep{Lai03}, the Monte Carlo maximum likelihood method of \citet{Gey94}, and variational inference \citep{Ble17, Tsi08}.



\subsection{Stochastic Approximation}

Stochastic approximation (SA) is a method originally proposed by \citet{Rob51} for finding roots of functions which can only be evaluated with noise. This method was expanded upon rapidly by, for example, \citet{Rob51} into a method for derivative-free optimization, and by \citet{Dvo56} with a systematic theoretical framework. Since the mid-20th century, SA methods have grown into a thriving research area. See, e.g., \citet{Kus97} or \citet{Bor22} for textbook-length treatments, and \citet{Lai03} for a survey paper which focuses on applications to statistics.

The core idea of SA is to iteratively update our estimate of the root, $\theta_*$, of some unobservable function, $\phi$, based on the value of a noisy realization of that function at the current estimate. Specifically, if $\theta_k$ is our estimate at iteration $k$ and $\hat{\phi} \approx \phi$, then our estimate at iteration $k+1$ is $\theta_{k+1} = \theta_k - \alpha_k \hat{\phi}(\theta_k)$, where $\alpha_k$ is a sequence which goes to zero at a particular rate. Since our sequence of weights goes to zero with $k$, the update terms become negligible in the limit and our estimate of $\theta_*$ stabilizes. The precise requirement for these weights is that $\sum_{k=1}^\infty \alpha_k = \infty$ and $\sum_{k=1}^\infty \alpha_k^2 < \infty$. A common choice is $\alpha_k = k^{-1}$. Numerous authors have studied convergence of the SA method in probability, almost surely and in $\mathcal{L}^1$ under various regularity conditions. See \citet{Lai03} for an excellent review of the history of stochastic approximation convergence theory.

There are a few specific applications of the general stochastic approximation which merit special attention. First, suppose that we want to optimize a function, $f$, which we can only evaluate approximately. Assume further that we are able to approximately evaluate the gradient of $f$, $\nabla f$. Setting $\phi = \nabla f$ and running the SA algorithm with $\hat{\phi} \approx \nabla f$ gives us a critical point for $f$. We can therefore use SA to optimize functions which cannot be evaluated exactly. This version of SA sees considerable use in the Machine Learning community under the name Stochastic Gradient Descent \citep{Bot10}.

Another specific instance of the SA method is when computing fixed points. For example, suppose that we want to find a point, $\theta_*$, such that $f(\theta_*) = \theta_*$. In this case, we set $\phi(\theta) = f(\theta) - \theta$ and run SA. The update equation in this case becomes a convex combination of the previous iteration's estimate and the current function value: $\theta_{k+1} = (1 - \alpha_k) \theta_k + \alpha_k \hat{f}(\theta_k)$. 

Several authors have investigated the use of SA with incomplete data. We refer to such an application as a stochastic approximation EM (SAEM) algorithm. Both cases of the SA method described above find use in constructing particular SAEM algorithms. An advantage of the SAEM algorithm over MCEM is that in SAEM we choose the Monte Carlo size once at the beginning and leave it fixed for every iteration. We can think of the method as automatically increasing the MC size since the estimate at each iteration is a weighted sum of all the estimates which came before it. A disadvantage of SAEM is that it requires us to select the sequence $\{ \alpha_k \}$, commonly referred to as the ``step size''. Choosing $\alpha_k$ too large will mean the algorithm takes a long time to stabilize, while choosing $\alpha_k$ too small causes the algorithm to stabilize before it reaches its limiting value (and will therefore take a long time to reach this limit). \citet{Jan06} gives some guidance on choosing this step size based on the goal of balancing bias with variance. This work also presents a convergence diagnostic based on the ideas of \citet{Caf05} which allows for a posteriori assessment of whether the step size was too small. 

A particular implementation of the SAEM method by \citet{Gu98I} suggests setting $\phi$ to the observed data score, $\phi(\theta) = S(\theta; y)$. This function can be estimated using Proposition \ref{thm:EM_decomp} (i) and the same Monte Carlo approach we use with MCEM. In the case of a vector-valued parameter, \citeauthor{Gu98I} also recommend pre-multiplying $\hat{\phi}$ by a matrix which converges to the inverse of the observed data information matrix. Such a sequence of matrices can be constructed using Louis' Identity (Proposition \ref{thm:info_decomp}) and Monte Carlo. Note that we use the same sample of missing data to update our estimates of the parameter and the observed data information matrix at each iteration. Similar work by \citet{Gu98II} extend the same SAEM construction to accommodate Markov Chain Monte Carlo sampling. \citet{Gu01} discuss how to apply the above methodology to the analysis of spatial models, and incorporate a second stage to the method in which estimates are averaged across iterations. This averaging has been found to improve performance of SAEM, and of stochastic approximation more generally \citep{Del99,Pol92}.

\citet{Jan06} presents an SAEM implementation in which $\phi(\theta)$ is the EM objective function (i.e., the conditional expectation of the complete data log-likelihood) minus $\theta$. Thus, the limit of such an algorithm is a fixed-point of the EM update. This ties convergence of \citeauthor{Jan06}'s method more closely to the convergence theory of the original EM algorithm \citep{Wu83}. In fact, \citeauthor{Jan06} leverages this connection to exploit the relationship between the EM algorithm's convergence rate and certain information matrices \citep{Men94}. This connection allows \citeauthor{Jan06} to estimate the bias and variance of the SAEM estimates, which are then used to calibrate the step size, $\alpha_k$.

A subtly different line of research on the SAEM method has been developed by a group at the National Institute for Research in Digital Science and Technology (INRIA) in France \citep[see, e.g.,][for a discussion of some of their methods]{Cel95}. The goal here is to augment the EM algorithm, rather than to facilitate the application of EM-type methods when ordinary EM is intractable. More precisely, methods from this group introduce a stochastic perturbation to the EM algorithm, with the goal of escaping fixed points which are locally, but not globally, optimal. Early work centered around a Stochastic EM (SEM) algorithm, which is equivalent to the MCEM algorithm with a Monte Carlo size of one \citep{Cel87,Cel95}. Later, they also propose a method which they refer to as SAEM (although it does not quite fit into our above framework), in which each iteration's parameter update is a convex combination of the EM and MCEM updates from the previous iteration's estimate. Here, as with the SEM algorithm, the MCEM update is computed with a Monte Carlo size of one \citep{Cel92, Cel95}.

\subsection{Monte Carlo Maximum Likelihood}




\subsection{Variational Methods}

\textbf{This section was written by memory (i.e. without looking anything up). The idea is to more efficiently get words on the page, then I can fix things later.}

Variational inference is a set of methods for approximating intractable densities \citep{need}. The literature on variational inference is vast. See \citet{Ble17} for a recent survey paper. \citet[Section 13.7]{Gel13} give a textbook-level overview focusing on applications to Bayesian inference. Numerous authors have discussed the connection between variational inference and the EM algorithm; see, e.g., \citet{Nea98,Tsi08}.

The central idea of variational inference is to frame the target density as the exact solution of a functional optimization problem, where the decision variable typically ranges over densities. The domain of functions is then restricted to some set which is easier to work with. Finally, optimization is performed over this restricted class of functions. The result of this optimization is then our approximation to the target density. \hl{This discussion doesn't make it clear whether our estimator is the optimizer or the optimal objective function.} 

More generally, the optimization problem described above may be just one in a sequence of problems which may form part of an iterative algorithm. Herein lies the connection to the EM algorithm. In the M-step of EM, we compute expectations with respect to a particular conditional distribution. To embed this procedure in the variational inference framework, we define an optimization problem whose solution is the same conditional distribution.

We now give some details. First, let $q$ be some probability density for the missing data, $X$. We can write the observed data log-likelihood as:
%
\begin{align}
    \ell(\theta; y) &= \log f_c(y, x; \theta) - \log f_m(y, x; \theta)\\
    &= \log \left[ \frac{f_c(y, x; \theta)}{q(x)} \right] - \log \left[ \frac{f_m(y, x; \theta)}{q(x)} \right]\\
    &= \bQ \log \left[ \frac{f_c(y, X; \theta)}{q(X)} \right] - \bQ \log \left[ \frac{f_m(y, X; \theta)}{q(X)} \right] \label{eq:variational_expectation}\\
    &=: F(q, \theta) + \mathrm{KL}(q \rightarrow f_m(\theta)) \label{eq:variational_decomp}\\
    & \geq F(q, \theta)
\end{align}
%
where line (\ref{eq:variational_expectation}) holds because the left-hand side does not depend on $x$, and the last line follows from non-negativity of the KL divergence. The first term in line (\ref{eq:variational_decomp}) is called the ``evidence lower-bound'' (ELBO), as well as the ``variational free energy'', depending on the field of application \citep{need}. The latter name comes from an analogue in physics \citep{need}. The former name comes from applications to Bayesian inference, where the observed data log-likelihood can be seen as an evidence if we take $Y$ to be the data and $X$ to be the parameters. 

The EM algorithm can be re-framed as alternately maximizing $F$ with respect to $q$ and $\theta$. To see why, first recall that the KL divergence between two distributions is non-negative, and is zero if and only if the two distributions are equal. Starting at a fixed parameter value, $\theta_0$, maximizing $F$ with respect to $q$ is accomplished by setting $q(x) = f_m(y, x; \theta_0)$, as this minimizes the KL divergence in (\ref{eq:variational_decomp}) and the left-hand side is constant in $q$. The resulting ELBO is equal to $Q(\theta|\theta_0) + \xi$, where $Q$ is the EM objective function and $\xi$ is constant in $\theta$. Maximizing the ELBO in $\theta$ is therefore equivalent to maximizing the EM objective function. This maximization gives a new parameter value, which is used as input to the next iteration. See Theorem 1 of \citet{Nea98} for a more formal proof.

While the EM algorithm is obtained by maximizing $q$ over an unrestricted class of densities, other procedures can be formulated by restricting this class. One popular example is the ``mean-field'' approximation, which involves optimizing over the class of densities which factor over their arguments. That is, the class of functions $\mathcal{Q}_{MF}:=\{ \mathrm{Densities\ } q: q(X_1, \ldots, X_p) = \prod_{j=1}^p q_j(X_j) \}$. 

A major advantage of the mean-field approximation is that an iterative algorithm exists for finding the density, $q$, which maximizes the ELBO. This algorithm performs coordinate ascent, and the coordinate updates are closely related to computation of the full conditional distributions in Gibbs sampling \citep{need}. Write $q^{(k)} = \prod q_j^{(k)}$ for the current value of $q$, and $\bQ^{(k)}_{-j}$ for expectation with respect to all the missing variables except $j$, with distributions from $q^{(k)}$ (\hl{awk?}). The update formula is
%
\begin{align}
    q_j^{(k+1)} &\propto \exp \left[ \bQ^{(k)}_{-j} \ell_c(y, x_j, X_{-j}) \right] \label{eq:var_inf_update}
\end{align}
%
where $X_{-j}$ is all the missing variables other than $X_j$. See Section 2.4 of \citet{Ble17} for a derivation of (\ref{eq:var_inf_update}). The overall algorithm consists of repeatedly cycling through updating each coordinate's distribution until some convergence criterion is met (\hl{how is convergence assessed?}).

Note that, so far, our discussion of how to compute the mean-field approximate density for $X$ has not addressed $\theta$. To apply mean-field variational inference to EM-type problems, we substitute the mean-field density into the ELBO and maximize over $\theta$. This new value of $\theta$ is fed back into ($\ref{eq:var_inf_update}$), giving us a different complete data likelihood function and, hence, a new optimal density.


\section{Simulation}
\label{sec:simulation}

An obstacle to implementing the MCEM algorithm which was not addressed in Section \ref{sec:MCEM}, and which is also relevant to many of the methods described in Section \ref{sec:alternatives}, is how to generate the necessary Monte Carlo samples. It is in general a hard problem to simulate from arbitrary conditional distributions. In this section, we discuss several methods for simulating the required observations at each step of the MCEM algorithm. All the topics that we cover here have their own vast literature, which we cannot hope to cover in their entirety. We instead give only a brief overview, focusing on aspects which are particularly relevant to use with the MCEM algorithm, and direct the reader to other, more focused, reviews.

\subsection{Importance Sampling}

Broadly speaking, importance sampling is a framework for approximating intractable expectations. A typical use case is when we want to evaluate the expected value of some function, $\bF h$, and this expectation is not just analytically intractable, but the distribution $bF$ is impossible (or impractical) to sample from. The latter restriction prevents us from using ordinary Monte Carlo integration. Instead, we re-write $\bF h$ as $\bG [h \cdot (f/g)]$ for some new distribution, $\bG$, where $f$ and $g$ are the densities of $\bF$ and $\bG$ respectively. Provided that $\bG$ is easy to sample from, we can estimate this alternative expression for our target expectation via Monte Carlo integration with samples drawn from $\bG$. We call $\bF$ the target distribution and $\bG$ the proposal distribution.  

A classic reference on importance sampling and other Monte Carlo methods is the book by \citet{Rob04}; particularly Chapters 3 and 4. Chapter 8 of the book by \citet{Cho20} gives a more current overview of importance sampling, with a focus on its application to Sequential Monte Carlo. \citet{Elv22} give a survey of modern methods for extending the importance sampling framework, with an emphasis on two main approaches: multiple importance sampling and adaptive importance sampling. See \citet{Elv19} for a review of multiple importance sampling methods, and \citet{Bug17} for more on adaptive importance sampling. \citet{Aga17} give a survey paper level treatment of some more theoretical considerations for importance sampling.

In the rest of this section, we describe a few simple modifications which can ease implementation and improve performance when using importance sampling with the MCEM algorithm.

When using importance sampling with the MCEM algorithm, our target distribution is the missing data distribution. In some settings, this distribution may be difficult to describe exactly. However, the integrand is the (log-)likelihood for the complete data distribution, so it is reasonable to expect that we can evaluate this complete data density. From the definition of conditional probability, the missing data density is proportional to the complete data density, provided that we treat the latter as a function of the missing data and hold the observed data fixed. The proportionality constant here is the observed data density, so it is unlikely that we will be able to normalize the missing data density exactly (otherwise, we could just work directly with the observed data likelihood). 

A simple modification of importance sampling, which circumvents the need for normalized densities, is to compute the importance weights as normal, then normalize them to sum to one. This is referred to as ``auto-normalized'', or ``self-normalized'', importance sampling \citep[see, e.g.,][]{Elv22}. The reason self-normalized importance sampling works is that the unknown normalizing constant cancels in the numerator and denominator of our normalized weights. In fact, our proposal distribution can also be un-normalized, and the ratio of the two normalizing constants cancels when we normalize our weights.

There are, however, disadvantages of self-normalized importance sampling compared to the exact importance sampling which we can do when both target and proposal densities are known exactly. One important limitation is that our estimator of $\bF h$ is no longer unbiased, and is instead only asymptotically unbiased. In fact, as a ratio estimator, the standard error of a self-normalized importance sampling estimator can only be determined asymptotically. This is fine in problems where it is easy to sample from our proposal distribution, $\bG$, but in high dimensional problems for example, even simulating from $\bG$ may be costly, and more care must be taken with the discrepancy between asymptotic results and finite-sample behaviour.

It is well-known that the performance of an importance sampling estimator depends on how closely the proposal distribution matches the target \citep{Rob04}. One thing that can go wrong is if the importance weight (i.e.\ the likelihood ratio between these two distributions) does not have sufficiently many finite moments \citep{Aga17}. A simple modification to our importance weights which guarantees infinitely many finite moments is to specify a threshold value, and truncate any weights which fall above the threshold (i.e., set weights which fall above the threshold equal to the threshold value). This ``truncated importance sampling'' method is proposed and analyzed by \citet{Ion08}. Two strategies are proposed in this work for selecting the threshold: the first is to simply use the square root of the Monte Carlo size, $\sqrt{M}$, while the latter is based on unbiased risk estimation and gives a value better tailored to the specific problem. Note that these recommendations are based on exact importance sampling. If self-normalization is used, the general recommendation is instead to truncate at $\sqrt{M}$ times the mean of the un-normalized weights. Truncated importance sampling represents an example of the bias-variance trade-off. Truncating weights at some threshold reduces the variance of our importance sampling estimator, but also introduces bias. This trade-off highlights the importance of selecting an appropriate threshold value: too large and the variance reduction will be negligible, but too small and the bias will be unacceptable.

\citet{Veh22} propose an alternative method for handling large importance weights, called Pareto Smoothed Importance Sampling (PSIS). The idea of this method is to fit a Generalized Pareto Distribution to the largest importance weights, then replace those large weights with quantiles of the fitted distribution. One of the parameter's fitted values also serves as a useful diagnostic for how closely our proposal distribution matches the target. An advantage of PSIS over the truncated importance sampling method of \citet{Ion08} is that the bias here is smaller, although it is not clear in general which method has better mean-squared error \citep[see][for extensive numerical comparisons]{Veh22}.  

\hl{Further discuss application to MCEM? Illustrate application to MCEM?}

An approach proposed by \citet{Lev01} seeks to use importance sampling to save computing time when running the MCEM algorithm. This computational efficiency is especially important in their context where Markov Chain Monte Carlo sampling is used (see \ref{sec:MCMC}), so generating a single sample takes quite some time. Their idea is to generate a single sample from the missing data distribution with some reference value for $\theta$, $\theta_{ref}$. This sample is then re-used at every MCEM iteration, with conditional expectations under the current parameter value computed by taking importance ratios with respect to $\theta_{ref}$. This raises the question of whether a single proposal distribution can be adequate for every target distribution along the MCEM trajectory. To address this concern, \citeauthor{Lev01} suggest running MCEM for a few iterations with fresh importance samples and only drawing the sample which will be used for their method after a sufficiently long ``burn-in'' period (they suggest running for one minute). 

\subsection{Particle-Based Methods}

It might be worth cutting this section. I can describe SMC, but I can't say much about how it relates to MCEM besides being another Monte Carlo algorithm. That being said, the work of \citet{Gol16} is worth mentioning in the context of MCEM. Also, the work of \citet{Mof14}, which is cited by \citeauthor{Gol16}. I will need to re-read both of these and look into who they are cited by.

\subsection{Markov Chain Monte Carlo}
\label{sec:MCMC}

The core idea of Markov Chain Monte Carlo (MCMC) sampling is to construct a Markov Chain from which we can simulate, and which has stationary distribution equal to the target distribution. Popular methods to construct such a Markov Chain are Gibbs sampling and the Metropolis-Hastings algorithm. See \citet{Gel13} or \citet{Rob04} for excellent textbook-length overviews. A popular implementation of MCMC sampling is the \texttt{Stan} programming language \citep{Sta22}, and its \texttt{R} interface, \texttt{RStan} \citep{Sta23}.

Both Gibbs sampling and \mh start with a random variable, $X = (X_1, \ldots, X_d)$, which we wish to simulate. Let $f$ be the density of $X$. These methods proceed by iteratively simulating draws of the vector $X$ from some distribution based on the draw from the previous iteration. 

Gibbs sampling\index{Gibbs Sampling} is based on successively sampling each component of $X$ conditional on all the other components. The order in which this conditioning is performed is a bit subtle, however. When generating $X_i$, we condition on the values of $X_1,\ldots, X_{i-1}$ from the current iteration and the values of $X_{i+1}, \ldots, X_d$ from the previous iteration. Once we reach the end of $X$, we start a new iteration.  More generally, we can group the components of $X$, and simulate an entire group conditional on the others, provided that we follow the structure outlined above \hl{(awk)}.

The Metropolis-Hastings algorithm\index{Metropolis-Hastings Algorithm} closely resembles rejection sampling. We start each iteration by generating a candidate value of $X$ from some proposal distribution (this distribution may depend on the previous iteration's value of $X$). Write $J(x | x_0)$ for the proposal density, where $x_0$ is the value of $X$ from the previous iteration. Next, we define an acceptance probability, $r := f(x) J(x_0 | x) / f(x_0) J(x | x_0)$, and accept the proposed value of $X$ with probability $r \wedge 1$. With probability $(1-r) \vee 0$, we reject the proposed $x$ and instead set the current iteration's value to $x_0$. We then proceed to the next iteration. Note that rejecting still adds an observation to our Monte Carlo sample, this value just happens to be identical to the one proceeding it.

\hl{I'm not super comfortable with this material. I want to acknowledge it and use it to motivate the more complicated adjustments to make MCEM work with MCMC, but I also don't want to say anything silly. I'm also going to need references.} Limit theory for estimators based on MCMC sampling are more complicated than the similar theory for estimators based on iid or importance sampling (\hl{SMC?}). This additional complexity stems from the dependence between draws from an MCMC sampler. While convergence for iid and importance sampling can be established using the Law of Large Numbers and the classical Central Limit Theorem, similar results for MCMC sampling make use of the Ergodic Theorem and Markov Chain Central Limit Theorem. A challenge in implementing the Markov Chain Central Limit Theorem is that the asymptotic variance depends on pairwise covariances between points in the chain which are arbitrarily far apart. Estimation of this asymptotic variance is therefore challenging in practice. 

\citet{Lev01} propose a method to simplify the analysis of estimates based on MCMC sampling. Their approach consists of subsampling the original chain using Poisson spacings with increasing mean. The result is that a target function averaged over the subsampled chain satisfies a classical Central Limit Theorem (i.e. one with no covariance terms in the asymptotic variance). We can estimate the mean and standard error of our subsample estimator using the entire chain and construct confidence intervals for the mean of the target function. \citeauthor{Lev01} apply this strategy for constructing confidence intervals along the same lines as \citet{Boo99}. Here, we construct a confidence interval for the gradient in the EM objective function, and increase the Monte Carlo size if the confidence interval for the current iteration contains the subsample estimate from the previous iteration. \hl{There's still something here I don't understand. What parameter is this CI for? In the paper, the CI is given by Equation 2.8 and justified by Theorem 1. A problem I'm having is that the thing which Theorem 1 asserts is asymptotically normal does not contain any parameters.} This work is further extended to provide uncertainty quantification for the MCEM update based on MCMC sampling by \citet{Lev04}. Furthermore, \citeauthor{Lev04} give a principled argument for how much to increase the Monte Carlo sample size when doing so is deemed necessary \hl{something about two MCEM updates. I didn't follow closely, but I'm not sure that it's critical to do so}.



\newpage

\begin{appendices}
    \section{Likelihood for Gene Frequency Estimation}
    \label{app:blood}

    In this appendix, we present details for the analysis of our example of estimating gene frequency. See Section \ref{sec:eg-genes} for formulation of the model and definition of notation.

    \subsection{Observed Data Likelihood, Score and Information}
    \label{app:blood_obs}

    Let $\pi_i$ be the probability of blood type $i$. The observed data likelihood for our model can be written as follows:
    %
    \begin{align}
        \ell(\theta; y) &= \log \begin{pmatrix} n \\ y \end{pmatrix} + \sum y_i \log \pi_i(\theta)\\
        & \equiv \sum y_i \log \pi_i\\
        &\equiv 2 y_1 \log r + y_2 \log(p^2 + 2pr) + y_3 \log(q^2 + 2qr) + y_4 \log pq
    \end{align}
    where we use $\equiv$ to denote equality up to additive constants which do not depend on $\theta$.

    Differentiating $\ell$ wrt $\theta$ gives the observed data score.
    %
    \begin{align}
        S(\theta; y) &= \begin{pmatrix}
            \partial_p \ell(\theta; y)\\
            \partial_q \ell(\theta; y) 
        \end{pmatrix} \mathrm{, where}\\
        \partial_p \ell(\theta; y) &= - \frac{2 y_1}{r}  + \frac{2r y_2}{p^2 + 2pr}  - \frac{2q y_3}{q^2 + 2qr}  + \frac{y_4}{p} \label{eq:gene_obs_score1}\\
        \partial_q \ell(\theta; Y) &= - \frac{2 y_1}{r}  - \frac{2p y_2}{p^2 + 2pr}  + \frac{2r y_3}{q^2 + 2qr}  + \frac{y_4}{q} \label{eq:gene_obs_score2}
    \end{align}
    %
    Solving the score equation, $S(\theta) = 0$, thus reduces to solving a system of two polynomials in $p$ and $q$. Since $p$ and $q$ are proportions, we reject any roots outside the unit simplex.

    Differentiating $\ell$ again and multiplying by $-1$ gives the observed data information matrix. To simplify notation, let $p_y = p^2 + 2pr$ and $q_y = q^2 + 2qr$.
    %
    \begin{align}
        I(\theta;y) &= - \begin{bmatrix}
            \partial^2_p \ell(\theta; y) & \partial_{p,q} \ell(\theta; y)\\
            \partial_{p,q} \ell(\theta; y) & \partial^2_q \ell(\theta; y)
        \end{bmatrix} \mathrm{, where}\\
        \partial^2_p \ell(\theta; y) &=  \frac{2y_1}{r^2} + \frac{2 y_2 (p_y + 2r^2)}{p_y^2} + \frac{4 y_3 q^2}{q_y^2} + \frac{y_4}{p^2}\\
        \partial_{p,q} \ell(\theta; y) &=  \frac{2y_1}{r^2} + \frac{2 y_2 p^2}{p_y^2} + \frac{2 y_3 q^2}{q_y^2}\\
        \partial^2_q \ell(\theta; y) &=  \frac{y_1}{r^2} + \frac{4 y_2 p^2}{p_y^2} + \frac{2 y_3 (q_y + 2r)}{q_y^2} + \frac{y_4}{q^2}
    \end{align}
    %
    The asymptotic standard error of our MLE is $I^{-1}$, evaluated at the estimate.

    \subsection{Complete Data Likelihood, Score and Information}
    \label{app:blood_complete}

    The complete data distribution for our model can be written as follows. Write $\rho_i$ for the probability of genotype $i$. See Table \ref{tab:blood_type_complete} for the values of these probabilities.
    %
    \begin{align}
        \ell_c(\theta; y,x) &= \log \begin{pmatrix} n \\ x \end{pmatrix} + \sum x_i \log \rho_i(\theta)\\
        & \equiv \sum y_i \log \rho_i\\
        &\equiv 2 x_1 \log r + x_2 \log pr + 2 x_3 \log p + x_4 \log qr + 2 x_5 \log q + x_6 \log pq\\
        &= (2 x_1 + x_2 + x_4) \log r + (x_2 + 2 x_3 + x_6) \log p + (x_4 + 2 x_5 + x_6) \log q\\
        &= n_O \log r + n_A \log p + n_B \log q
    \end{align}
    %
    where $n_O$, $n_A$ and $n_B$ are the number of times allele O, A and B arise respectively in the sampled genotypes. Note that $\ell_c$ depends on $y$ only through $x$, so we suppress $y$ from our notation for complete data quantities. The complete data score function is
    %
    \begin{align}
        S_c(\theta; x) &= \begin{pmatrix}
            \partial_p \ell_c(\theta; x)\\
            \partial_q \ell_c(\theta; x) 
        \end{pmatrix} \mathrm{, where}\\
        \partial_p \ell_c(\theta; x) &= \frac{x_2 + 2 x_3 + x_6}{p} - \frac{2x_1 + x_2 + x_4}{r} = \frac{n_A}{p} - \frac{n_O}{r} \label{eq:comp_score1}\\
        \partial_p \ell_c(\theta; x) &= \frac{x_4 + 2 x_5 + x_6}{q} - \frac{2x_1 + x_2 + x_4}{r} = \frac{n_B}{q} - \frac{n_O}{r} \label{eq:comp_score2}
    \end{align}
    %
    Notice that the score is linear in $x$. To make this relationship explicit, we write $S_c(\theta; x) = M(\theta) x$, where $M(\theta) \in \bR^{2 \times 6}$ consists of the coefficients on $x$ in (\ref{eq:comp_score1}) and (\ref{eq:comp_score2}). We will make use of this linearity later.

    Next, we give the information matrix for the complete data.
    %
    \begin{align}
        I_c(\theta;x) &= - \begin{bmatrix}
            \partial^2_p \ell_c(\theta; x) & \partial_{p,q} \ell_c(\theta; x)\\
            \partial_{p,q} \ell_c(\theta; x) & \partial^2_q \ell_c(\theta; x)
        \end{bmatrix} \mathrm{, where}\\
        \partial^2_p \ell_c(\theta; x) &=  \frac{x_2 + 2 x_3 + x_6}{p^2} + \frac{2x_1 + x_2 + x_4}{r^2} = \frac{n_A}{p^2} + \frac{n_O}{r^2}\\
        \partial_{p,q} \ell_c(\theta; x) &=   \frac{2x_1 + x_2 + x_4}{r^2} = \frac{n_O}{r^2}\\
        \partial^2_q \ell_c(\theta; x) &=  \frac{x_4 + 2 x_5 + x_6}{q^2} + \frac{2x_1 + x_2 + x_4}{r^2} = \frac{n_B}{q^2} + \frac{n_O}{r^2}
    \end{align}
    %
    

    \subsection{Missing Data Distribution}
    \label{app:blood_miss}

    Many quantities which arise in the EM and MCEM algorithms depend on the missing data distribution (i.e.\ the conditional distribution of $X$ given $Y=y$). This distribution is best described componentwise in $X$. First, note that $X_1$ and $X_6$ occur in $Y$, so we have $X_1 = y_1$ and $X_6 = y_4$. Next, we have that $X_2 + X_3 = y_2$ and $X_4 + X_5 = y_3$. Thus, we can write $X_2 \sim \mathrm{Bin}(y_2, 2pr / (p^2 + 2pr))$ and $X_4 \sim \mathrm{Bin}(y_3, 2qr / (q^2 + 2qr))$. Finally, we recover $X_3$ and $X_5$ by subtracting $X_2$ from $y_2$ and $X_4$ from $y_3$ respectively.

    We make frequent require the first few conditional moments of $X$. We list them here for convenience. Let $\alpha_1 = 2pr / (p^2 + 2pr)$ be the probability parameter for the binomial distribution of $X_2$ given $Y$, and $\alpha_2 = 1 - \alpha_1$. Similarly, let $\beta_1 = 2qr / (q^2 + 2qr)$ correspond to $X_4$ and $\beta_2 = 1 - \beta_1$. \hl{Notation in this section is new. I'm not sure I will use it.}
    %
    \begin{align}
        \bE(X | Y=y) &= \begin{pmatrix}
            y_1,  y_2 \alpha_1,  y_2 \alpha_2,  y_3 \beta_1,  y_3 \beta_2,  y_4
        \end{pmatrix}^T\\
        &=: \mu_m\\
        \bV(X | Y=y) &= \begin{pmatrix}
            0 & 0 & 0 & 0 & 0 & 0\\
            0 & y_2 \alpha_1 \alpha_2 & - y_2 \alpha_1 \alpha_2 & 0 & 0 & 0\\
            0 & - y_2 \alpha_1 \alpha_2 & y_2 \alpha_1 \alpha_2 & 0 & 0 & 0\\
            0 & 0 & 0 & y_3 \beta_1 \beta_2 & - y_3 \beta_1 \beta_2 & 0\\
            0 & 0 & 0 & -y_3 \beta_1 \beta_2 & y_3 \beta_1 \beta_2 & 0\\
            0 & 0 & 0 & 0 & 0 & 0
        \end{pmatrix}\\
        &=: \Sigma_m\\
        \bE(XX^T | Y=y) &= \Sigma_m + \mu_m \mu_m^T
    \end{align}
    %
    Conditional expectations of the number of alleles of each kind will be of particular interest.
    %
    \begin{align}
        \nu_O & := \bE(n_O|y)\\
         &= 2y_1 + \frac{y_2 pr}{p^2 + 2pr} + \frac{y_3 qr}{q^2 + 2qr}\\
        &= 2y_1 + y_2 \left( \frac{\rho_2}{\rho_2 + \rho_3} \right) + y_3 \left( \frac{\rho_4}{\rho_4 + \rho_5} \right) &\left( = 2y_1 +  y_2 \left( \frac{\rho_2}{\pi_2} \right) + y_3 \left( \frac{\rho_4}{\pi_3} \right) \right)\\
        \nonumber \\
        \nu_A & := \bE(n_A|y)\\
        &= \frac{2 y_2 pr}{p^2 + 2pr} + \frac{2y_2 p^2}{p^2 + 2pr} + y_4\\
        &= y_2 \left( \frac{\rho_2}{\rho_2 + \rho_3} + 2 \frac{\rho_3}{\rho_2 + \rho_3} \right) + y_4 &\left(= y_2 \left( \frac{\rho_2}{\pi_2} + 2\frac{\rho_3}{\pi_2} \right) + y_4 \right)\\
        &= y_2 \left( 1 + \frac{p^2}{p^2 + 2pr} \right) + y_4\\
        \nonumber \\
        \nu_B &:= \bE(n_B|y)\\
        &= \frac{2 y_3 qr}{q^2 + 2qr} + \frac{2y_3 q^2}{q^2 + 2qr} + y_4\\
        &= y_3 \left( \frac{\rho_4}{\rho_4 + \rho_5} + 2\frac{\rho_5}{\rho_4 + \rho_5} \right) + y_4 &\left(= y_3 \left( \frac{\rho_4}{\pi_3} + 2\frac{\rho_5}{\pi_3} \right) + y_4 \right)\\
        &= y_3 \left( 1 + \frac{q^2}{q^2 + 2qr} \right) + y_4
    \end{align}


    \subsection{EM Algorithm}

    In order to apply the EM algorithm, we must construct and optimize the EM objective function. That is, we must compute $Q(\theta|\theta_0) = \bE_{\theta_0} \left[ \ell_c(\theta; y, X) | Y=y \right]$. The conditional distribution of $X$ given $Y=y$ is best described componentwise in $X$. First, note that $X_1$ and $X_6$ occur in $Y$, so we have $X_1 = y_1$ and $X_6 = y_4$. Next, we have that $X_2 + X_3 = y_2$ and $X_4 + X_5 = y_3$. Thus, we can write $X_2 \sim \mathrm{Bin}(y_2, 2pr / (p^2 + 2pr))$ and $X_4 \sim \mathrm{Bin}(y_3, 2qr / (q^2 + 2qr))$. Finally, we recover $X_3$ and $X_5$ by subtracting $X_2$ from $y_2$ and $X_4$ from $y_3$ respectively.
    
    Before writing out the EM objective function, we first compute conditional expectations of the sample allele counts.
    %
    \begin{align}
        \bE(n_O|y) &= 2y_1 + \frac{y_2 pr}{p^2 + 2pr} + \frac{y_3 qr}{q^2 + 2qr}\\
        &= 2y_1 + y_2 \left( \frac{\rho_2}{\rho_2 + \rho_3} \right) + y_3 \left( \frac{\rho_4}{\rho_4 + \rho_5} \right) &\left( = 2y_1 +  y_2 \left( \frac{\rho_2}{\pi_2} \right) + y_3 \left( \frac{\rho_4}{\pi_3} \right) \right)\\
        &=: \nu_O\\
        \nonumber \\
        \bE(n_A|y) &= \frac{2 y_2 pr}{p^2 + 2pr} + \frac{2y_2 p^2}{p^2 + 2pr} + y_4\\
        &= y_2 \left( \frac{\rho_2}{\rho_2 + \rho_3} + 2 \frac{\rho_3}{\rho_2 + \rho_3} \right) + y_4 &\left(= y_2 \left( \frac{\rho_2}{\pi_2} + 2\frac{\rho_3}{\pi_2} \right) + y_4 \right)\\
        &= y_2 \left( 1 + \frac{p^2}{p^2 + 2pr} \right) + y_4\\
        &=: \nu_A\\
        \nonumber \\
        \bE(n_B|y) &= \frac{2 y_3 qr}{q^2 + 2qr} + \frac{2y_3 q^2}{q^2 + 2qr} + y_4\\
        &= y_3 \left( \frac{\rho_4}{\rho_4 + \rho_5} + 2\frac{\rho_5}{\rho_4 + \rho_5} \right) + y_4 &\left(= y_3 \left( \frac{\rho_4}{\pi_3} + 2\frac{\rho_5}{\pi_3} \right) + y_4 \right)\\
        &= y_3 \left( 1 + \frac{q^2}{q^2 + 2qr} \right) + y_4\\
        &=: \nu_B
    \end{align}
    
    
    The EM objective function can be written as
    %
    \begin{align}
		Q(\theta | \theta_0) &:= \bE_{\theta_0} [\ell_c (\theta; X) | Y=y]\\
		&\equiv \nu_O^{(0)} \log r + \nu_A^{(0)} \log p + \nu_B^{(0)} \log q
	\end{align}
    %
    where a superscript zero denotes that the quantity is computed by taking an expectation under $\theta_0$. Maximizing $Q$ analytically with respect to $\theta$ gives the following expression for the EM update:
    %
    \begin{align}
        M(\theta_{k-1}) &= \begin{pmatrix}
            \hat{p}_k\\ \hat{q}_k
        \end{pmatrix}\\
        &= \begin{pmatrix}
            \nu_A^{(k-1)} / 2n\\
            \nu_B^{(k-1)} / 2n
        \end{pmatrix}
    \end{align}
    %
    where a superscript $k-1$ denotes that the quantity is computed by taking an expectation under $\theta_{k-1}$. Note that the equation for a stationary point of the EM algorithm, $M(\theta) = \theta$, has identical roots to the observed data score equation, $S(\theta) = 0$ \hl{(confirm this)}.

    \subsubsection{Asymptotic Standard Error}

    Recall that the EM algorithm computes the MLE, which has asymptotic covariance matrix equal to the inverse Fisher information matrix evaluated at the true parameter value. In practice, we estimate this covariance with the inverse of the observed information matrix evaluated at the MLE. Using Proposition \ref{thm:info_decomp}, we can calculate the observed information matrix using conditional expectations of quantities derived from the complete data likelihood. 
    
    To this end, we need to evaluate the conditional expectations in expression (\ref{eq:info_at_MLE}) of Proposition \ref{thm:info_decomp}. It will be convenient for us to write $S_c(\theta) =: \mathscr{S}(\theta) X$. Then 
    %
    \begin{align}
        \mathcal{I}_c(\hat{\theta}) &= \begin{bmatrix}
            \frac{\nu_A}{p^2} + \frac{\nu_O}{r^2} & \frac{\nu_O}{r^2}\\
            \frac{\nu_O}{r^2} & \frac{\nu_B}{q^2} + \frac{\nu_O}{r^2}
        \end{bmatrix} \mathrm{, and}\\
        \bE_{\hat{\theta}} [ S_c(\hat{\theta}) S_c(\hat{\theta})^T | Y=y] &= \mathscr{S}(\hat{\theta}) \bE_{\hat{\theta}} \left[ X X^T | Y=y \right] \mathscr{S}(\hat{\theta}) \\
        &= \mathscr{S}(\hat{\theta}) (\Sigma_m + \mu_M \mu_M^T) \mathscr{S}(\hat{\theta})\\
    \end{align}
    %
    While it is possible to expand the above expressions, they quickly become too long to easily interpret. We instead leave these as computational formulas and use them as a guide for writing \texttt{R} or \texttt{Julia} code.


    \subsection{Monte Carlo EM}


\end{appendices}

\newpage

\textbf{Check for ``Citation Needed'' before publishing.}

\bibliographystyle{plainnat}
\bibliography{mybib}

\printindex
\end{document}  

